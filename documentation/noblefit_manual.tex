\documentclass[12pt]{article}
\usepackage[utf8]{inputenc}
\usepackage[british]{babel}
\usepackage[margin=3.4cm,a4paper]{geometry}
\usepackage{times} \usepackage{mathptmx}
\usepackage[T1]{fontenc}
\usepackage{booktabs}
\usepackage{natbib}
\usepackage[parfill]{parskip}
\usepackage{microtype}
\usepackage{color} %red, green, blue, yellow, cyan, magenta, black, white
\usepackage[colorlinks=true,urlcolor=blue,linkcolor=blue,citecolor=blue,filecolor=blue]{hyperref}
\usepackage{listings}
   \lstset{language=Matlab,
           basicstyle=\ttfamily\small,
           keywordstyle=\ttfamily,
          breaklines=true
          }
\definecolor{mygreen}{RGB}{28,172,0} % color values Red, Green, Blue
\definecolor{mylilas}{RGB}{170,55,241}

\newcommand{\noblefit}{{\tt NOBLEFIT}}
\newcommand{\secref}[1]{Sec.~\ref{sec:#1}}


\title{\Huge \noblefit\ \\[1ex]
\Large A Matlab/\mbox{GNU Octave} toolbox to fit (environmental) data}
\author{Matthias S.\ Brennwald\\ \url{matthias.brennwald@eawag.ch}
}
\date{\today}

\begin{document}


\maketitle
% \clearpage
\tableofcontents \clearpage

\phantom{x}\vfill
{\footnotesize
\noblefit\  is free software; you can redistribute it and/or modify it under the terms of the GNU General Public License as published by the Free Software Foundation; either version 2 of the License, or (at your option) any later version.  \noblefit\  is distributed in the hope that it will be useful, but WITHOUT ANY WARRANTY; without even the implied warranty of MERCHANTABILITY or FITNESS FOR A PARTICULAR PURPOSE.  See the GNU General Public License for more details.  You should have received a copy of the GNU General Public License along with \noblefit\ ; if not, write to the Free Software Foundation, Inc., 51 Franklin St, Fifth Floor, Boston, MA  02110-1301 USA\par
Copyright (C) 2014 Matthias S. Brennwald, Eawag (Swiss Federal Institute of Aquatic Science and Technology)\par
% ALREADY GIVEN ABOVE Contact: matthias.brennwald@eawag.ch
}
\clearpage

\section{What is \noblefit?}
\noblefit\ is a flexible Matlab/\mbox{GNU Octave} toolbox for quantitative interpretation of (environmental) tracers in terms of environmental processes and models (e.g., dissolved noble gases, other atmospheric gases, or just about anything else that deserves quantitative model-based interpretation).\par

In contrast to similar tools\citep{Aeschbach:1999} where the possible tracers and models are hard-wired into the code, \noblefit\ is designed to allow the user to define his own tracers and models.

\section{Installation and Setup}
This text assumes you are familiar with Matlab or GNU Octave, and that you have a working installation of either Matlab or GNU Octave on your computer.

\subsection{Download \noblefit}\label{sec:get_mpic}
There are two ways to get \noblefit: 
\begin{itemize}
	\item The easy method is to download \noblefit\ as a ZIP archive, and expand the files from the archive. You can download the ZIP archive from \url{http://sourceforge.net/p/noblefit/code/HEAD/tarball}.
	\item Alternatively, if you have subversion (SVN) software installed on your computer, you can get \noblefit\ as a SVN repository, which makes upgrading \noblefit\ very easy. You can checkout the SVN repository from \url{http://sourceforge.net/p/noblefit/code/HEAD/tree}.
\end{itemize}

\subsection{Install \noblefit}\label{sec:install_mpic}
Once you have a copy of \noblefit\ downloaded to your computer, you should move the \noblefit\ folder to a convenient location on your computer. I like to keep all your Matlab/\mbox{GNU Octave} code and packages in one directory, which contains several subdirectories for the different packages and projects. This greatly helps Matlab/\mbox{GNU Octave} to find your files. For instance, I keep all my m-files in {\tt \textasciitilde{}/m-files/}, so \noblefit\ goes to {\tt \textasciitilde{}/m-files/noblefit}. Then I tell Matlab/GNU Octave where to look for the \noblefit\ functions by including the command {\tt addpath('\textasciitilde{}/m-files/noblefit')} in the {\tt startup.m} file (Matlab) or the {\tt .octaverc} file (GNU Octave).

\section{Working with \noblefit}
% The \noblefit\ tools work the same as any other Matlab/\mbox{GNU Octave} tool. First you need to make sure Matlab or GNU Octave knows where to look for the \noblefit\ files (\secref{install_mpic}). Then you just type the name of the functions to the command shell to execute them.\par

\subsection{Overview}
The general approach of the \noblefit\ package is to fit a model to a given data set using the $\chi^2$ regression method \citep{Press:1986}. \noblefit\ reports the best-fit values of the fitted model variables, their standard errors (calculated by propagation of the data errors), and the statistics about the goodness of the fit ($\chi^2$, $p$ value, and degrees of freedom of the regression).\par

Before a model can be fitted to a data set, the fitting problem must be defined as follows ($n$: number of data values per data set, $m$: number of fitted model variables; $ n > m$):
\begin{itemize}
\item The user must provide the measured (or observed) data to \noblefit. Each data set consists of the observed values $\vec{y} = (y_1, \ldots, y_n)$ and the associated standard errors $\vec{e} = (e_1, \ldots, e_n)$ (note that data values without errors are useless). In simple cases, the data can be entered directly on the Matlab/\mbox{GNU Octave} terminal. If the data set is large, or if there are many different data sets (e.g., data from different samples), it may be more convenient and reliable to load the data from a file.
\item The user must provide the model function $\vec{Y} = F(\vec{X})$ to \noblefit. This is done by writing a Matlab/\mbox{GNU Octave} function that calculates the modelled data values $\vec{Y} = y_1, \ldots, y_n$ as a function of the input variables $\vec{X} = X_1, \ldots, X_m$ of the model.
\item The user must provide the inital values of the fitted model variable. These initial values will be used as a starting point to find the best-fit values.
\item If only a subset of the model variables is used in the fit, the user must provide the values of the model parameters that will be used to evaluate the model (i.e., the fixed values of the model variables that are not fitted).
\end{itemize}

Once the fitting problem is defined, the {\tt noblefit.m} function is called to fit the model to the data using the $\chi^2$ regression method \citep{Press:1986}. To this end, {\tt noblefit.m} finds the values of the fitted model variables that minimize the sum of the squares of the error-weighted residuals between the predicted and the measured data ($\chi^2$):
\begin{displaymath}
\chi^2(\vec{X}) = \sum_1^n \left(\frac{Y_i - y_i}{e_i}\right)^2 = \sum_1^n \left(\frac{F_i(\vec{X}) - y_i}{e_i}\right)^2
\end{displaymath}

Once {\tt noblefit.m} found the $\chi^2$ minimum, it reports the best-fit values of the fitted model variables $\vec{X}$, their errors ($\vec{E} = E_1,\ldots,E_m$; calculated by propagation of the data errors), and the some information about the goodness of the fit ($\chi^2$, $p$ value, and degrees of freedom of the regression).\par

The call to the {\tt noblefit} function looks like this:\\
{\tt [X,E,chi2,DF,p] = noblefit (F,x,m,vf,v0,vs,vmin,vmax)}
\begin{itemize}
\item Input arguments (see \secref{inputargs} for details):
\begin{itemize}
	\item {\tt F}: model function
	\item {\tt x}: measured data
	\item {\tt m}: measured variables included in the fit
	\item {\tt vf}: index to the fitted model variables
	\item {\tt v0}: initial values of fit variables and values of fixed model variables
	\item {\tt vs}: scaling factors of model variables (optional)
	\item {\tt vmin}: minimum values of fitted model variables
	\item {\tt vmax}: maximum values of fitted model variables
\end{itemize}
\item Output arguments (see \secref{outputargs} for details):
\begin{itemize}
	\item {\tt X}: best fit values of fitted model variables
	\item {\tt E}: standard errors of X
	\item {\tt chi2}: $\chi^2$ value
	\item {\tt DF}: degrees of freedom of the fit
	\item {\tt p}: $p$ value of fit
\end{itemize}
\end{itemize}
\par

\subsection{Input arguments}\label{sec:inputargs}
\subsubsection{Model function}
work in progress.

\subsubsection{Measured data}
work in progress.

\subsubsection{Measured variables included in the fit}
work in progress.

\subsubsection{Index to fitted model variables}
work in progress.

\subsubsection{Initial values of fit variables and values of fixed model variables}
work in progress.

\subsubsection{Scaling factors of model variables}
work in progress.

\subsubsection{Minimum values of fitted model variables}
work in progress.

\subsubsection{Maximum values of fitted model variables}
work in progress.


\subsection{Output arguments}\label{sec:outputargs}

\subsubsection{Best fit values of fitted model variables}
work in progress.

\subsubsection{Standard errors of best fit values}
work in progress.

\subsubsection{$\chi^2$ value}
work in progress.

\subsubsection{Degrees of freedom of the fit}
work in progress.

\subsubsection{$p$ value of the fit}
work in progress.

\section{Worked examples}
There are a few complete examples of how to use \noblefit\ in the the {\tt examples} folder. While I am working to improve and expand this manual, I'd recommend you take a look at those worked examples to get started.


\section{\noblefit\  tools reference}\label{sec:tools_ref}
This section lists all \noblefit\ functions and describes their functionality.

\subsection{nf\_atmos\_gas}\label{sec:ref-nf_atmos_gas}
\begin{lstlisting}
 [C_atm,v_atm,D,M_mol,H] = nf_atmos_gas (gas,T,S,p_atm,year,hemisphere)

 Returns dissolved gas concentrations in air-saturated water, volumetric gas content in dry air and molecular diffusivity in water.
 Concentrations are calculated as gas amount (ccSTP) per mass of water (g) at temperature T and salinity S

 INPUT:
 T:        temperature of water in deg. C
 S:        salinity in per mille (g/kg)
 p_atm:	total atmospheric air pressure, including water vapour (hPa, which is the same as mbar)
 gas:	    'He', 'He-3', 'He-4' (after Weiss)
           'RHe' (3He/4He ratio)
           'Ne', 'Ne-20', 'Ne_20','Ne-22', 'Ne_22' (after Weiss, isotope fractionation from Beyerle)
           'Ar', 'Ar-36', 'Ar_36', 'Ar-40', 'Ar_40' (after Weiss, isotope fractionation from Beyerle)
           'Kr', 'Kr-78', 'Kr_78', 'Kr-80', 'Kr_80', 'Kr-82', 'Kr_82', 'Kr-84', 'Kr_84', 'Kr-86', 'Kr_86 (after Weiss)
           'Xe', 'Xe-124', 'Xe_124', 'Xe-126', 'Xe_126', Xe-128', 'Xe_128', 'Xe-129', 'Xe_129', 'Xe-130', 'Xe_130', 'Xe-131', 'Xe_131', 'Xe-132', 'Xe_132', 'Xe-134', 'Xe_134', 'Xe-136', 'Xe_136' (after Clever)
           'SF6'
           'CFC11', 'CFC12', 'CFC113'
           'O2', 'O2-34', 'O2-35', 'O2-36'
           'N2', 'N2-28, 'N2-29', 'N2-30'
 year:     year of gas exchange with atmosphere (calendar year, with decimals; example: 1975.0 corresponds to 1. Jan of 1975.098 corresponds to 5. Feb. 1975, etc.). This is only relevant for those gases with time-variable partial pressures in the atmosphere (e.g. CFCs, SF6)
 hemisphere: string indicating hemisphere (one of 'north', 'south', or 'global'). If the hemisphere argument is not specified, hemisphere = 'global' is used.

 OUTPUT:
 C_atm:	concentration in air-saturated water (ccSTP/g)
 v_atm:	volume fraction in dry air
 D: 	    molecular diffusivities (m^2/s)
 M_mol:    molar mass of the gas (g/mol), values taken from http://www.webqc.org/mmcalc.php
 H:        Henrys Law coefficient in (hPa/(ccSTP/g)), as in p* = H * C_atm, where p* is the partial pressure of the gas species in the gas phase

 EXAMPLES:
 1. To get the Kr ASW concentration (ccSTP/g) in fresh water (temperature = 7.5 deg.C) at atmospheric pressure of 991 hPa:
 [C_atm,v_atm,D,M_mol,H] = nf_atmos_gas ('Kr',7.5,0,991); C_atm

 2. To get the SF6 ASW concentration (ccSTP/g) in mid-1983 northern hemisphere in fresh water (temperatures of 0-10 deg.C, salinity 6 g/kg) at atmospheric pressure of 983 hPa:
 [C_atm,v_atm,D,M_mol,H] = nf_atmos_gas ('SF6',[0:10],6,983,1983.5,'north'); C_atm

 
\end{lstlisting}

\subsection{nf\_objfun}\label{sec:ref-nf_objfun}
\begin{lstlisting}
 G = nf_objfun (PF,P0,PFmin,PFmax,mdl,X_val,X_err)

 Objective function for mimization in parameter regression. Determines the modelled values using the model for the given parameter values, and calculates the chi^2 value from the difference to the data values. If called with fit parameter values that exceed the limits in PFmin and PFmax, the chi^2 value will be calculated such that the chi^2 minimizer will look elsewhere for a 'better' optimum (see commented code for details).

 INPUT:
 PF: vector of fitted model parameter values
 P0: vector of constant model parameter values
 PFmin: vector of minimum values allowed for the fitted parameters
 PFmax: vector of maximum values allowed for the fitted parameters
 mdl: string containing call to the model function using PF and P0
 X_val: values of observed/measured data (rows correspond to samples, columns correspond to one tracers)
 X_err: standard errors of X_val

 OUTPUT:
 G: chi^2 value

 
\end{lstlisting}

\subsection{nf\_read\_datafile}\label{sec:ref-nf_read_datafile}
\begin{lstlisting}
 [data,tracers] = nf_read_datafile (file,options);

 Reads data from a formatted text file. The data needs to be organized in columns as follows:
 - Columns are assumed to be separated by tabs. Other delimiters may be specified using 'options'.
 - The first line must be a header line with names of the data in the columns.
 - The first column must contain sample names (treated as string).
 - The remaining columns must contain the data values (either numbers, NA, NaN, or empty).
 - If column titles include units (or anything else) in parentheses, the parentheses part is removed from the name (it may be useful to have the units in the data file, but the unit will be in the way for data formatting)
 - Data columns containing the data uncertainties (errors) are identified by adding 'err' somehere in the title. Example: if the Ne concentrations are given in column with title 'Ne', the column title of the corresponding errors could be 'Ne err', 'err. Ne', 'Ne_err', etc.

 INPUT:
 file: file name, may include path to file (string)
 options (optional): struct to provide options. May be useful to provide details about the format of the data file, may be useful to specify special file formats (e.g. ouptut from 4D database or input files for Franks noble fitter. Known options:
 - options.replace_zeros: if set, replace data values equal to zero by opt_replace_zeros (scalar)
 - options.filter_InvIsotopeRatios: if set to non-zero value (scalar), try to make sure that isotope ratios are such that the low-mass isotope is divided by the high-mass isotope (e.g., replace 40Ar/36Ar by 36Ar/40Ar).

 OUTPUT:
 data: array of structs with data values and corresponding errors. Every struct corresponds to one line in the data file. Fieldnames correspond to the column titles in the header line. Sample names are stored in field 'name'.

 
\end{lstlisting}

\subsection{noblefit}\label{sec:ref-noblefit}
\begin{lstlisting}
 [par_val,par_err,chi2,DF,pVal,cov,res] = noblefit (model,tracer_data,tracers,par_usage,par0,par_norm,par_range,par_min,par_max);

 Frontend to fit a given model to observed data using chi^2 regression.

 INPUT:
 model: model name (string). More specifically, this is the name of the function that returnscan either be the name of one of the standard models provided with gasfit, or a custom model function provided by the user (function name must be on the search path).
 tracer_data: either the name of an ASCII file containing the data (with column headers that correspond to the tracer names) or a struct variable containing the observed data (either a single struct corresponding to one sample, or a vector of structs for more than one sample). The tracer names must correspond to those in the model function.
 tracers: a cell string containing the names of the tracers that are to be used in the fit (names must correspond to those used by the model function).
 par_usage: vector indicating usage of the model parameter values. par_usage is of the same format as the vector used as the input argument of the model function. Values are as follows:
   - par_usage(i) = 0: The i-th model parameter is used as a constant in the minimization problem, i.e., it is not optimized during fitting of the model to the data.
   - par_usage(i) = 1: the i-th parameter is optimized to obtain the best model fit for each individual sample.
   - other values may be implemented in a later version (e.g., for ensemble fits of a model parameter to an ensemble of data from multiple samples)
 par0: parameter values used for the regression (vector of the same format as used for the input argument of the model function). Depending on the parameter usage (see par_usage), the values are used as fixed values or initial values used in the minimization problem.
 par_norm (optional): typical scale of variation of the parameter values, used to normalise fitting parameters during model fitting (vector or matrix of sime size as par0, values used for fit parameters must not be zero). Note that the scaling factors reflect the RANGE OF VARIATION, not the absolute value. For instance, if infiltration date is somewhere between 1950 and 2013, a suitable scaling factor would be 10, not 1000.
 par_min (optional): min allowed parameter values for fit (vector or matrix of same size as par0, only values for fitted parameters will be used). Values may be -Inf to indicate no limits. Note: fit results may sightly exceed the limits, if the best fit value is close to the limit. This is due to the way the limits are treated in the fitting routine. The effect should be small, but please make sure the values are ok for you.
 par_max (optional): max allowed parameter values for fit (vector or matrix of same size as par0, only values for fitted parameters will be used). Values may be Inf to indicate no limits. Note: fit results may sightly exceed the limits, if the best fit value is close to the limit. This is due to the way the limits are treated in the fitting routine. The effect should be small, but please make sure the values are ok for you.

 OUTPUT:
 par_val: best-fit estimates of the parameter values (vector of the same format as used for the input argument of the model function).
 par_err: standard errors of par_val (vector).
 chi2: chi2 of fit
 DF: degrees of freedom of fit ( = number of data points - number of fitted model parameters)
 pVal: p-value of chi2, as given from cumulative chi2-density function with DF degrees of freedom: pVal = 1 - chi2cdf(chi2,DF)
 cov: covariance matrices of the fits (cell array of matrices)
 res: error-weighted residuals of observed values (X) of sample k relative to modeled values (M), normalised by the standard error (E) of the observed values, i.e.: res(k,i) = (X_i - M_i) / E_i, where i is an index to the corresponding tracer (res is a matrix, each row corresponds to one sample, columns correspond to 'tracers')

 EXAMPLES:
 see files in Examples folder.

 
\end{lstlisting}



\bibliography{MB_Literatur}
\bibliographystyle{plain}

\end{document}

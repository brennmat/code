\subsection{nf\_atmos\_gas}\label{sec:ref-nf_atmos_gas}
\begin{lstlisting}
 [C_atm,v_atm,D,M_mol,H] = nf_atmos_gas (gas,T,S,p_atm,year,hemisphere)

 Returns dissolved gas concentrations in air-saturated water, volumetric gas content in dry air and molecular diffusivity in water.
 Concentrations are calculated as gas amount (ccSTP) per mass of water (g) at temperature T and salinity S

 INPUT:
 T:        temperature of water in deg. C
 S:        salinity in per mille (g/kg)
 p_atm:	total atmospheric air pressure, including water vapour (hPa, which is the same as mbar)
 gas:	    'He', 'He-3', 'He-4' (after Weiss)
           'RHe' (3He/4He ratio)
           'Ne', 'Ne-20', 'Ne_20','Ne-22', 'Ne_22' (after Weiss, isotope fractionation from Beyerle)
           'Ar', 'Ar-36', 'Ar_36', 'Ar-40', 'Ar_40' (after Weiss, isotope fractionation from Beyerle)
           'Kr', 'Kr-78', 'Kr_78', 'Kr-80', 'Kr_80', 'Kr-82', 'Kr_82', 'Kr-84', 'Kr_84', 'Kr-86', 'Kr_86 (after Weiss)
           'Xe', 'Xe-124', 'Xe_124', 'Xe-126', 'Xe_126', Xe-128', 'Xe_128', 'Xe-129', 'Xe_129', 'Xe-130', 'Xe_130', 'Xe-131', 'Xe_131', 'Xe-132', 'Xe_132', 'Xe-134', 'Xe_134', 'Xe-136', 'Xe_136' (after Clever)
           'SF6'
           'CFC11', 'CFC12', 'CFC113'
           'O2', 'O2-34', 'O2-35', 'O2-36'
           'N2', 'N2-28, 'N2-29', 'N2-30'
 year:     year of gas exchange with atmosphere (calendar year, with decimals; example: 1975.0 corresponds to 1. Jan of 1975.098 corresponds to 5. Feb. 1975, etc.). This is only relevant for those gases with time-variable partial pressures in the atmosphere (e.g. CFCs, SF6)
 hemisphere: string indicating hemisphere (one of 'north', 'south', or 'global'). If the hemisphere argument is not specified, hemisphere = 'global' is used.

 OUTPUT:
 C_atm:	concentration in air-saturated water (ccSTP/g)
 v_atm:	volume fraction in dry air
 D: 	    molecular diffusivities (m^2/s)
 M_mol:    molar mass of the gas (g/mol), values taken from http://www.webqc.org/mmcalc.php
 H:        Henrys Law coefficient in (hPa/(ccSTP/g)), as in p* = H * C_atm, where p* is the partial pressure of the gas species in the gas phase

 EXAMPLES:
 1. To get the Kr ASW concentration (ccSTP/g) in fresh water (temperature = 7.5 deg.C) at atmospheric pressure of 991 hPa:
 [C_atm,v_atm,D,M_mol,H] = nf_atmos_gas ('Kr',7.5,0,991); C_atm

 2. To get the SF6 ASW concentration (ccSTP/g) in mid-1983 northern hemisphere in fresh water (temperatures of 0-10 deg.C, salinity 6 g/kg) at atmospheric pressure of 983 hPa:
 [C_atm,v_atm,D,M_mol,H] = nf_atmos_gas ('SF6',[0:10],6,983,1983.5,'north'); C_atm

 
\end{lstlisting}

\subsection{nf\_objfun}\label{sec:ref-nf_objfun}
\begin{lstlisting}
 G = nf_objfun (PF,P0,PFmin,PFmax,mdl,X_val,X_err)

 Objective function for mimization in parameter regression. Determines the modelled values using the model for the given parameter values, and calculates the chi^2 value from the difference to the data values. If called with fit parameter values that exceed the limits in PFmin and PFmax, the chi^2 value will be calculated such that the chi^2 minimizer will look elsewhere for a 'better' optimum (see commented code for details). Note that many techniques for fitting with constrained parameter ranges rely on mapping the "infinite chi2 surface" to the allowed parameter range. This distorts the chi2 surface, and while the minimum chi2 value will correspond to the correct best-fit paramter values, the "distorted" chi2 value cannot be statistically interpreted in the same way as the "undistorted" chi2. nf_objfun.m therefore avoids distorting the chi2 surface in the allowed parameter range, and the resulting chi2 value can be interpreted using the conventionel chi2 statistics.

 INPUT:
 PF: vector of fitted model parameter values
 P0: vector of constant model parameter values
 PFmin: vector of minimum values allowed for the fitted parameters
 PFmax: vector of maximum values allowed for the fitted parameters
 mdl: string containing call to the model function using PF and P0
 X_val: values of observed/measured data (rows correspond to samples, columns correspond to one tracers)
 X_err: standard errors of X_val

 OUTPUT:
 G: chi^2 value

 
\end{lstlisting}

\subsection{nf\_read\_datafile}\label{sec:ref-nf_read_datafile}
\begin{lstlisting}
 [data,tracers] = nf_read_datafile (file,options);

 Reads data from a formatted text file. The data needs to be organized in columns as follows:
 - Columns are assumed to be separated by tabs. Other delimiters may be specified using 'options'.
 - The first line must be a header line with names of the data in the columns.
 - The first column must contain sample names (treated as string).
 - The remaining columns must contain the data values (either numbers, NA, NaN, or empty).
 - If column titles include units (or anything else) in parentheses, the parentheses part is removed from the name (it may be useful to have the units in the data file, but the unit will be in the way for data formatting)
 - Data columns containing the data uncertainties (errors) are identified by adding 'err' somehere in the title. Example: if the Ne concentrations are given in column with title 'Ne', the column title of the corresponding errors could be 'Ne err', 'err. Ne', 'Ne_err', etc.

 INPUT:
 file: file name, may include path to file (string)
 options (optional): struct to provide options. May be useful to provide details about the format of the data file, may be useful to specify special file formats (e.g. ouptut from 4D database or input files for Franks noble fitter. Known options:
 - options.replace_zeros: if set, replace data values equal to zero by opt_replace_zeros (scalar)
 - options.filter_InvIsotopeRatios: if set to non-zero value (scalar), try to make sure that isotope ratios are such that the low-mass isotope is divided by the high-mass isotope (e.g., replace 40Ar/36Ar by 36Ar/40Ar).

 OUTPUT:
 data: array of structs with data values and corresponding errors. Every struct corresponds to one line in the data file. Fieldnames correspond to the column titles in the header line. Sample names are stored in field 'name'.

 
\end{lstlisting}

\subsection{noblefit}\label{sec:ref-noblefit}
\begin{lstlisting}
 [par_val,par_err,chi2,DF,pVal,cov,res] = noblefit (model,tracer_data,tracers,par_usage,par0,par_norm,par_range,par_min,par_max);

 Frontend to fit a given model to observed data using chi^2 regression.

 INPUT:
 model: model name (string). More specifically, this is the name of the function that returnscan either be the name of one of the standard models provided with gasfit, or a custom model function provided by the user (function name must be on the search path).
 tracer_data: either the name of an ASCII file containing the data (with column headers that correspond to the tracer names) or a struct variable containing the observed data (either a single struct corresponding to one sample, or a vector of structs for more than one sample). The tracer names must correspond to those in the model function.
 tracers: a cell string containing the names of the tracers that are to be used in the fit (names must correspond to those used by the model function).
 par_usage: vector indicating usage of the model parameter values. par_usage is of the same format as the vector used as the input argument of the model function. Values are as follows:
   - par_usage(i) = 0: The i-th model parameter is used as a constant in the minimization problem, i.e., it is not optimized during fitting of the model to the data.
   - par_usage(i) = 1: the i-th parameter is optimized to obtain the best model fit for each individual sample.
   - other values may be implemented in a later version (e.g., for ensemble fits of a model parameter to an ensemble of data from multiple samples)
 par0: parameter values used for the regression (vector of the same format as used for the input argument of the model function). Depending on the parameter usage (see par_usage), the values are used as fixed values or initial values used in the minimization problem.
 par_norm (optional): typical scale of variation of the parameter values, used to normalise fitting parameters during model fitting (vector or matrix of sime size as par0, values used for fit parameters must not be zero). Note that the scaling factors reflect the RANGE OF VARIATION, not the absolute value. For instance, if infiltration date is somewhere between 1950 and 2013, a suitable scaling factor would be 10, not 1000.
 par_min (optional): min allowed parameter values for fit (vector or matrix of same size as par0, only values for fitted parameters will be used). Values may be -Inf to indicate no limits. Note: fit results may sightly exceed the limits, if the best fit value is close to the limit. This is due to the way the limits are treated in the fitting routine. The effect should be small, but please make sure the values are ok for you.
 par_max (optional): max allowed parameter values for fit (vector or matrix of same size as par0, only values for fitted parameters will be used). Values may be Inf to indicate no limits. Note: fit results may sightly exceed the limits, if the best fit value is close to the limit. This is due to the way the limits are treated in the fitting routine. The effect should be small, but please make sure the values are ok for you.

 OUTPUT:
 par_val: best-fit estimates of the parameter values (vector of the same format as used for the input argument of the model function).
 par_err: standard errors of par_val (vector).
 chi2: chi2 of fit
 DF: degrees of freedom of fit ( = number of data points - number of fitted model parameters)
 pVal: p-value of chi2, as given from cumulative chi2-density function with DF degrees of freedom: pVal = 1 - chi2cdf(chi2,DF)
 cov: covariance matrices of the fits (cell array of matrices)
 res: error-weighted residuals of observed values (X) of sample k relative to modeled values (M), normalised by the standard error (E) of the observed values, i.e.: res(k,i) = (X_i - M_i) / E_i, where i is an index to the corresponding tracer (res is a matrix, each row corresponds to one sample, columns correspond to 'tracers')

 EXAMPLES:
 see files in Examples folder.

 
\end{lstlisting}

